A análise das bibliotecas mais usadas que fornecem mecanismos de resiliência permitiu concluir que não existem bibliotecas que suportem \textit{KMP}.
A título de exemplo, a biblioteca \textit{Resilience4j}~\cite{resilience4j} que foi desenhada para \textit{Java}, já providencia um módulo de interoperabilidade com \textit{Kotlin}, mas apenas exclusivamente para a \textit{JVM}.
Por esse motivo, aplicações \textit{KMP} que necessitem de mecanismos de resiliência têm de escolher entre:

\begin{itemize}[topsep=0pt,itemsep=0pt,partopsep=0pt, parsep=0pt]
    \item recorrer a bibliotecas que fornecem mecanismos de resiliência e que são específicas para cada \textit{target}, o que aumenta a complexidade e a redundância do código;
    \item implementar a sua própria solução, o que aumenta, principalmente, o tempo de desenvolvimento.
\end{itemize}
