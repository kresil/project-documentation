\section{Organization Description}\label{sec:organization-description}

The project is organized in a GitHub organization to facilitate collaboration and management of the project's repositories.
The organization, which hosts the project's repositories, is named \textbf{Kresil} and is available in this~\ulhref{https://https://github.com/kresil}{link},
As the organization is private, access is restricted to the organization's members.
One can request access to the organization by contacting the organization's owners.

The organization is divided into repositories, each with a specific purpose as outlined in Table~\ref{tab:organization-repositories}.

\begin{table}[!htb]
    \centering
    \caption{Organization Repositories}\label{tab:organization-repositories}
    \vspace{0.3cm}
    \begin{tabular}{|l|p{10cm}|l|}
        \hline
        \textbf{Name}   & \textbf{Description}                                                                                              & \textbf{Url}                                    \\ \hline
        \textbf{Kresil} & Contains the main project's codebase, including the Kotlin Multiplatform library and modules for Ktor integration & \ulhref{https://github.com/kresil/kresil}{link}                                                  \\ \hline
        \textbf{Experiments} & Contains experiments exploring the Kotlin Multiplatform development, the Ktor framework and the Resilience4j library
        & \ulhref{https://github.com/kresil/experiments}{link}                                                       \\ \hline
        \textbf{Ps} &
        Outlines the founding proposal and technical documentation of the project & \ulhref{https://github.com/kresil/ps}{link}                                                       \\ \hline
    \end{tabular}
\end{table}


\section{Management}\label{sec:management}

The project management is done using GitHub Projects and Issues.
The project board is available in this~\ulhref{https://github.com/orgs/kresil/projects/1}{link},
where the project's tasks are organized in columns according to their status.
The board is constantly updated by the project's members, so that it reflects the current state of the project in any point in time.

In the development process,
the project's issues are used to track the tasks that need to be done
(e.g., bugs, features, etc.)
which are usually linked to a specific milestone.
The issues themselves are linked to a specific pull request (PR),
which is used to merge the changes into the default branch when the task or tasks are considered resolved.
As such, when accessing the project's repositories,
one can see the issues and pull requests that are currently open,
which provide insights into the project's current problems and attempts to solve them, respectively.


\section{Documentation}\label{sec:documentation}

The project documentation relevant to the Project and Seminary course is available in the \textbf{ps} repository,
and includes the report, proposal, presentations and poster, among other relevant documents.

All repositories have a \texttt{README.md} file that contains information about the repository structure, how to build and run the project, and other relevant information.
