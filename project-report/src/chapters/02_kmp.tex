\chapter{Kotlin Multiplatform}\label{ch:kotlin-multiplatform}

This chapter provides an overview of the technology, including its architecture, template structure, and other aspects relevant to the development of a multiplatform library.
The Kotlin Multiplatform (KMP) technology allows developers to share code across multiple platforms, such as Android and iOS for mobile applications, and/or JVM, JavaScript and Native for multiplatform overall.


\section{Architecture}\label{sec:architecture}

A KMP project is divided into three main categories of code:

\begin{itemize}
    \item \textbf{Common}: Code shared between all platforms (i.e., \textit{CommonMain, CommonTest});
    \item \textbf{Intermediate}: Code that can be shared on a subset of platforms (i.e., \textit{AppleMain, AppleTest});
    \item \textbf{Specific}: Code specific to a target platform (i.e., \texttt{\textit{<Platform>Main}}, \texttt{\textit{<Platform>Test}}).
\end{itemize}


An example of a KMP project architecture can be seen in Figure~\ref{fig:kmp-architecture}, but note that both \textit{Intermediate} and \textit{Specific} categories are optional.

\begin{figure}[!htb]
    \centering
    \includegraphics[width=0.8\textwidth]{../figures/02_kmp-architecture}
    \caption{Example of a KMP project architecture.}
    \label{fig:kmp-architecture}
\end{figure}

\subsection{Template}\label{subsec:template}

It is possible to create a KMP project from scratch, but it is recommended to use a template to facilitate the project's setup and configuration.
The official Kotlin Multiplatform template~\cite{kmp-github-template} provides a project structure
that includes the necessary configurations for building,
testing,
and deploying a multiplatform library for most platforms.

\subsection{Gradle Tasks}\label{subsec:available-gradle-tasks}

Gradle is a build automation tool for multi-language software development.
Offers support for all phases of a build process including compilation, verification, dependency resolving, test execution, source code generation, packaging and publishing~\cite{gradle}.

In a Gradle build using Kotlin DSL (domain-specific language), a project's configuration is primarily defined in two key files:

\begin{itemize}
    \item \textbf{build.gradle.kts} - Defines a project's build configuration.
    In KMP projects, this file is used to define a project's targets, dependencies in the respective source sets, and additional configurations if needed;
    \item \textbf{settings.gradle.kts} - Defines a project's structure and modules it contains.
\end{itemize}

In Gradle, tasks are the smallest unit of work that can be executed and are used to perform specific actions.
The templated uses the KMP plugin which includes several pre-configured Gradle tasks to facilitate building, testing,
and managing a project across multiple platforms configured as targets in the respective \textit{build gradle} file.
Some of the key tasks are:

\begin{itemize}
    \item \textbf{Build}: Compiles and assembles the project;
    \item \textbf{AllTests}: Runs the test cases for all platforms.
    To run platform-specific tests, use the \texttt{<Platform>Test} task (e.g., \textit{jvmTest});
    \item \textbf{Check}: Performs various checks on the project, including running tests and performing additional operations (e.g., linting, code analysis);
    \item \textbf{Clean}: Deletes the build directory, allowing for a clean build (i.e., not using cached artifacts).
\end{itemize}

A Gradle project can be organized into multiple subprojects, each with its own build file, settings and tasks.

\subsection{GitHub Actions}\label{subsec:github-actions}

The template provides workflows for GitHub Actions~\cite{github-actions}, which are used to automate tasks such as building, testing, and deploying a project.
The configurations for GitHub Actions are located in the \textit{.github/workflows} folder, and include the following workflows:

\begin{itemize}
    \item \textbf{gradle.yml} - Builds and tests the project using Gradle against some of the available platforms.
    Runs on push and pull request git events to the default branch;
    \item \textbf{deploy.yml} - deploys the library artifacts to a repository in Maven Central~\cite{maven-central}, following a pre-defined authentication configuration.
\end{itemize}

\subsection{Folder Structure}\label{subsec:folder-structure}

The template is organized into several folders, each serving a specific purpose.
Below is a brief description of each folder, with an emphasis on the \textit{src} folders:

\begin{itemize}
    \item \textbf{.github/}: Contains configurations for GitHub Actions, which are used to automate tasks;
    \item \textbf{gradle/}: Contains configuration files and scripts related to the Gradle build system.
    This folder typically includes:
    \begin{itemize}
        \item \textbf{wrapper/}: Contains the wrapper files and configurations, which standardizes a project on a given Gradle version for more reliable and robust builds~\cite{gradle-wrapper};
        \item \textbf{libs.versions.toml}: Defines the versions of the libraries and plugins used in a project, as dependencies, in a centralized manner (regularly known as version catalog~\cite{gradle-version-catalog}).
    \end{itemize}
    \item \textbf{convention-plugins/}: Encapsulates and reuses common build logic across multiple Gradle projects or modules (e.g., for publishing, testing, etc.);
    \item \textbf{library/}: Contains the source code for the library and the build configuration file.
    \begin{itemize}
        \item \textbf{src/}: Contains the source code for the library, divided into multiple submodules based on the target platforms:
        \begin{itemize}
            \item \textbf{commonMain/}: Contains the common code shared across all platforms;
            \item \textbf{jvmMain/}: Contains the source code specific to the JVM platform;
            \item \textbf{jsMain/}: Contains the source code specific to the JavaScript platform;
            \item \textbf{iosMain/}: Contains the source code specific to the iOS platform;
            \item \textbf{androidMain/}: Contains the source code specific to the Android platform;
            \item And all of these module counterparts for the test code (e.g., \textit{commonTest, jvmTest, jsTest, etc}).
        \end{itemize}
        \item \textbf{build.gradle.kts}: Defines the targets, dependencies, and additional configurations for the library;
    \end{itemize}
    \item \textbf{build.gradle.kts}: The main build configuration file for the project, where the subprojects common configurations are defined;
    \item \textbf{settings.gradle.kts}: Configures the Gradle build settings for the project, including the root project name and module inclusion.
\end{itemize}

Based on the described template's project structure, the following structure was adopted for developing a KMP library:

\begin{itemize}
    \item \textbf{\texttt{<}kmp\_package\_name\texttt{>}}: name of the KMP library in root directory;
    \begin{itemize}
        \item \textbf{apps}: defines the modules that will consume the KMP library (e.g.,\textttt{js-app}, \texttt{android-app});
        \item \textbf{lib/shared}: defines the library's code to be shared by the consuming modules;
        \begin{itemize}
            \item \textbf{src}: defines the target submodules of the library including their test counterparts (i.e., \texttt{<Platform>Main}, \texttt{<Platform>Test});
            \item \textbf{build.gradle.kts}: defines the library's dependencies, targets, and additional configurations.
        \end{itemize}
    \end{itemize}
\end{itemize}


\section{Platform-Dependent Code}\label{sec:platform-dependent-code}

As code sharing across platforms is the primary objective of KMP, it should be written as platform-independently as possible (i.e., aggregating as much code in the hierarchically higher categories).
However, it is sometimes necessary to create specific code for a given platform, regularly referred to as \textit{target}, in the following situations:

\begin{itemize}
    \item Access to API's specific to the \textit{target} is required (e.g., \textit{Java's File API});
    \item The libraries available in the common category (i.e., \textit{Standard Kotlin Library}, libraries and extensions from \textit{Kotlinx}) do not cover the desired functionalities, and third-party libraries either do not support them or are avoided to reduce dependencies;
    \item When specific types and functions defined in Kotlin need to be accessed in other languages (e.g., \textit{Javascript}).
    As such, an adapter is required to communicate with the common category code,
    in Kotlin, from the native code of the \textit{target}, which can be defined in the \textit{Intermediate} or \textit{Specific} categories.
\end{itemize}

To create specific code for a \textit{target}, the mechanism \textit{expect/actual}~\cite{kmp-expect-actual} is used.
This mechanism allows defining the code to be implemented in an abstracted way and its implementation, respectively.


\section{Supported Targets}\label{sec:supported-targets}

The project supports the following targets:

\begin{itemize}
    \item \textbf{JVM}: Allows running the code on the \textit{Java Virtual Machine};
    \item \textbf{JavaScript}: Allows running the code in a browser or \textit{Node.js} environment;
    \item \textbf{Android}: Allows running the code on Android devices;
    \item \textbf{Native}: Allows running the code on platforms that support \textit{Kotlin/Native}, excluding macOS and iOS, because the lack of access to the necessary hardware for testing.
\end{itemize}


\section{Kotlin Library Practices}\label{sec:kotlin-library-practices}

The library follows the best practices for developing a Kotlin library, including:

\begin{itemize}
    \item Adherence to \href{https://kotlinlang.org/docs/coding-conventions.html#coding-conventions-for-libraries}{Kotlin Coding Conventions for Libraries} to ensure consistency and readability;
    \item Compliance with \href{https://kotlinlang.org/docs/api-guidelines-introduction.html}{Kotlin API Guidelines} to create a user-friendly and maintainable API.
\end{itemize}


\section{Multiplatform Considerations}\label{sec:multiplatform-considerations}

When developing a multiplatform library, it was necessary to consider the following aspects, which may be harder to manage in a multiplatform context:

\begin{itemize}
    \item \textbf{Concurrency}: Techniques and tools used to manage concurrency, ensuring thread safety and performance across platforms;
    \item \textbf{Time Representation and Measurement}: Strategies for representing and measuring time consistently across platforms;
    \item \textbf{Logging}: Implementation of logging to facilitate debugging and monitoring;
    \item \textbf{Mocking}: Strategies for mocking dependencies in tests to isolate and verify the behavior of individual components.
\end{itemize}
