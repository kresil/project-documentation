\chapter{Conclusion}\label{ch:conclusion}

The development of this project has focused
on addressing an important need
for building reliable and robust distributed systems in the Kotlin Multiplatform ecosystem, which is the lack of a dedicated library for resilience mechanisms.
This document explained the challenges and limitations of implementing such mechanisms in a multiplatform environment,
and the design choices that led to the chosen solution.

This research project aimed to develop a Kotlin Multiplatform library
that provides essential resilience mechanisms for Kotlin Multiplatform projects.
The Ktor integration was a later addition to the project, as of the time of writing, is
the only Kotlin Multiplatform framework for developing server and client services.
To integrate the mechanisms into Ktor, we developed plugins that leverage Ktor's extensibility interface.

The project began with a review of the existing solutions on resilience mechanism implementations,
in several platforms and languages, in order to identify the common patterns and practices, as well as the differences and limitations.
With that knowledge,
a model was designed to represent the common interface for all implemented resilience mechanisms by the library -
the Mechanism Model.

We explored various resilience mechanisms such as Retry, Circuit Breaker and Rate Limiter
detailing their functionality, configuration capabilities and default values associated with each policy,
implementation challenges and decisions, and integration into the Ktor framework as plugins.

In summary,
this project successfully developed a dedicated Kotlin Multiplatform library
that provides essential resilience mechanisms,
and provides a Ktor integration to leverage these mechanisms in server and client services.
The integration with Ktor not only validated the library's implementation but also provided immediate utility in a specific and widely-used context.
Comprehensive testing and documentation ensure the reliability and ease of adoption for developers.


\section{Future Work}\label{sec:future-work}

The library provides a starting point for building reliable and robust distributed systems in the Kotlin Multiplatform ecosystem, but there is still room for improvement and expansion.

The following are some possible future work that can be done to enhance the library:

\begin{itemize}
    \item \textbf{Additional Resilience Mechanisms}: The library can be expanded with additional resilience mechanisms such as Bulkhead,
    Timeout, and Fallback, and the corresponding Ktor plugins for client and server sides.
    This will complete the set of considered essential mechanisms for building resilient systems;
    \item \textbf{Registry}:
    Implement a registry system
    to access and manage the resilience mechanisms and their configurations at runtime and in a centralized manner;
    \item \textbf{Metrics}:
    Implement a metrics system to collect information of the resilience mechanisms execution (e.g., the number of retries, the number of failures, etc.) for monitoring and analysis purposes;
    \item \textbf{Testing Improvements}:
    Enhance the testing suites with more comprehensive and complex test cases
    to ensure the reliability and robustness of the library, as well as to cover more edge cases and scenarios.
    Extend the testing to cover the Ktor plugins as well;
    \item \textbf{Dedicated Pipeline}:
    Implement a pipeline mechanism,
    where multiple mechanisms can be attached in a specific order to handle different scenarios (i.e., easily combine two or more mechanisms such as retry and circuit breaker);
    \item \textbf{Documentation and Examples}:
    Improve the documentation
    and provide more examples to help developers understand and use the library effectively.
    A website or a dedicated page should be created
    to host these resources.
    A documentation engine such as Dokka~\cite{dokka} should be used to generate an additional documentation API website;
    \item \textbf{Adapter for JavaScript}: Implement an adapter to call the library from JavaScript code.
    As the current implementation of the library is only usable in Kotlin code, an adapter is needed to make it accessible from JavaScript code.
\end{itemize}
