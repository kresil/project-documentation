\chapter{Conclusion}\label{ch:conclusion}

The development of this project has focused
on addressing an important need
for building reliable and robust distributed systems in the Kotlin Multiplatform ecosystem, which is the lack of a dedicated library for resilience mechanisms.
This document explained the challenges and limitations of implementing such mechanisms in a multiplatform environment,
and the design choices that led to the chosen solution.

This research project aimed to develop a Kotlin Multiplatform library
that provides essential resilience mechanisms for Kotlin Multiplatform projects.
The Ktor integration was a later addition to the project, as of the time of writing, is
the only Kotlin Multiplatform framework for developing server and client services.
To integrate the mechanisms into Ktor, we developed plugins that leverage Ktor's extensibility interface.

The project began with a review of the existing solutions on resilience mechanism implementations,
in several platforms and languages, in order to identify the common patterns and practices, as well as the differences and limitations.
With that knowledge,
a model was designed to represent the common interface for all implemented resilience mechanisms by the library -
the Mechanism Model.

We explored various resilience mechanisms such as Retry, Circuit Breaker and Rate Limiter
detailing their functionality, configuration capabilities and default values associated with each policy,
implementation challenges and decisions, and integration into the Ktor framework as plugins.

In summary,
this project successfully developed a dedicated Kotlin Multiplatform library
that provides essential resilience mechanisms,
and provides a Ktor integration to leverage these mechanisms in server and client services.
The integration with Ktor not only validated the library's implementation but also provided immediate utility in a specific and widely-used context.
Comprehensive testing and documentation ensure the reliability and ease of adoption for developers.


\section{Future Work}\label{sec:future-work}

The library provides a starting point for building reliable and robust distributed systems in the Kotlin Multiplatform ecosystem, but there is still room for improvement and expansion.

The following are some possible future work that can be done to enhance the library in no particular order:

\begin{itemize}
    \item \textbf{Additional Resilience Mechanisms}:
    The library can be expanded with additional resilience mechanisms such as Bulkhead,
    Cache, Timeout, and Fallback.
    This will complete the set of considered essential mechanisms for building resilient systems.
    Additionally, implement the respective Ktor plugins for these mechanisms;
    \item \textbf{Registry}:
    Implement a registry system
    to access and manage the resilience mechanisms and their configurations at runtime and in a centralized manner;
    \item \textbf{Metrics}:
    Implement a metrics system to collect information of the resilience mechanisms execution (e.g., the number of retries, the number of failures, etc.) for monitoring and analysis purposes which could provide, among other benefits, insights on how to adjust the configuration of the mechanisms to improve the system's resilience and responsiveness;
    \item \textbf{Testing Improvements}:
    Enhance the testing suites with more comprehensive, complex and concurrent test cases
    to ensure the reliability and robustness of the library, as well as to cover more edge cases and scenarios.
    Use a dedicated Kotlin Multiplatform testing framework such as Kotest~\cite{kotest}
    to improve the maintainability and readability of the future developed tests.
    Tests that involve benchmarking and performance evaluation can also be added to measure the overhead of the mechanisms,
    and provide insights on how to optimize them.
    Extend these tests to cover the Ktor plugins as well;
    \item \textbf{Dedicated Pipeline}:
    Implement a pipeline,
    where multiple mechanisms can be attached in a specific order to handle different scenarios (e.g., combine an outer retry with an inner rate limiter, to retry after a request is rejected due to rate limiting);
    \item \textbf{Adapter for JavaScript}: Implement an adapter to call the library from JavaScript code.
    As the current implementation of the library is only usable in Kotlin code, an adapter is needed to make it accessible from JavaScript code;
    \item \textbf{Chaos Engineering}: Besides unit, integration and possibly functional tests, implement a Chaos Engineering module to test the resilience of the system under controlled conditions.
    \item \textbf{Documentation and Examples}:
    Improve the documentation
    and provide more examples to help developers understand and use the library effectively.
    A website or a dedicated page should be created
    to host these resources.
    A documentation engine such as Dokka~\cite{dokka} could be used to generate an additional documentation API website.
    \item \textbf{CI/CD Optimization}:
    Optimize the CI/CD pipeline by integrating tools like Dependabot~\cite{github-dependabot} to receive notifications and updates on the dependencies used in the project and other tools to enhance the development workflow;
    \item \textbf{Selective Dependency Import}:
    Offer the option to import one or more mechanisms as dependencies, rather than requiring all of them.
    This will allow developers to include only the necessary parts of the library, enhancing modularity and reducing the application's footprint.
    The same approach should be applied to the respective Ktor plugins.
\end{itemize}
