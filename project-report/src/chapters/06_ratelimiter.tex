\chapter{Rate Limiter}\label{ch:rate-limiter}

\resilienceMechanismChapterIntro{Rate Limiter}{proactive}


\section{Introduction}\label{sec:rate-limiter-introduction}

The Rate Limiter is a proactive resilience mechanism
that aims to control the rate at which requests are made to a specific service or resource.
By imposing a limit on the number of requests within a given time period,
the Rate Limiter helps to prevent overloading services,
ensuring stability and availability.
Rate limiting \textit{\enquote{is often employed to stop bad bots from negatively impacting a website or application.
Bot attacks that rate limiting can help mitigate include brute force attacks, DoS (denial of service)
    and DDoS (distributed denial of service) attacks, and web scraping.}}~\cite{cloudflare-rate-limiting}

\subsection{Relation To The Throttling Mechanism}\label{subsec:rate-limiter-throttling}

Rate limiting and throttling are closely related concepts often used interchangeably,
since both mechanisms control the rate of incoming requests to protect services from being overwhelmed, but they have distinct differences.
Rate limiting is a broader concept that encompasses various strategies to control the rate of incoming requests, including rejecting requests that exceed the predefined limits.
Throttling, on the other hand, specifically refers to the process of slowing down the rate of requests rather than outright rejecting them.

Throttling is typically used to ensure that high-priority requests are served first by delaying less critical ones,
while rate limiting enforces strict limits on the number of requests over a given time period.
For example, a rate limiter might allow 100 requests per minute from a single IP address, while a throttling mechanism might slow down the request rate after the first 50 requests in a minute but still allow some through.
