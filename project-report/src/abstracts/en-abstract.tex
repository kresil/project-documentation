\chapter*{Abstract}\label{ch:abstract}

% purpose:
This project addresses a critical need in the Kotlin Multiplatform ecosystem,
by developing a dedicated fault-tolerance library.
Kotlin Multiplatform is a technology that enables developers to share code across diverse platforms promoting code reuse and maintainability.

% goal:
The primary goal is
to create a Kotlin Multiplatform library
offering essential resilience mechanisms (e.g., Retry, Circuit Breaker, Rate Limiter)
that alone or combined, prevent or mitigate the inevitable failures that occur in distributed systems,
which impact the availability and reliability of services.
Furthermore, the project aims to integrate these mechanisms into Ktor,
the only Kotlin Multiplatform framework for building asynchronous server and client services.

% methodology:
The project began with a review of existing solutions for resilience mechanisms in various platforms and languages
to identify common patterns and practices, as well as differences and limitations.
With this knowledge,
a model was designed to represent the common interface for all implemented resilience mechanisms by the library -
the Mechanism Model.
The project then explored each of the
various (but not all) resilience mechanisms,
detailing their functionality,
the chosen design and implementation details, configuration capabilities, default values associated with each policy, and their integration into the Ktor framework as plugins.

% conclusion:
As the project is amidst development, future work includes
expanding the library with additional resilience mechanisms
and further improving its testing and documentation.
This way providing developers with a comprehensive and reliable toolkit for building resilient distributed systems
in Kotlin Multiplatform.

\textbf{Keywords:} Kotlin Multiplatform; library; fault-tolerance; resilience; distributed systems; Ktor framework
