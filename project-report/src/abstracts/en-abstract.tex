\chapter*{Abstract}\label{ch:abstract}

% introduction:
This document addresses a need in the Kotlin Multiplatform ecosystem,
by developing a fault-tolerance library.
Kotlin Multiplatform is a technology that enables developers to share code across multiple platforms, promoting code reuse and maintainability.

% goal:
The primary goal is
to create a Kotlin Multiplatform library
offering essential resilience mechanisms (e.g., Retry, Circuit Breaker, Rate Limiter).
These mechanisms, alone or combined, prevent or mitigate the inevitable failures that occur in distributed systems,
which impact the availability and reliability of services.
Furthermore, the project aims to integrate these mechanisms into Ktor,
a Kotlin Multiplatform framework for building asynchronous server and client services.

% methodology:
The project began with a review of existing solutions in various platforms and languages to identify common patterns and practices, as well as differences and limitations.
With this knowledge,
a model was designed to represent the common interface for all implemented resilience mechanisms by the library - the Mechanism Model.
The project then explored each of the
various resilience mechanisms,
detailing their functionality,
the chosen design and implementation details, configuration capabilities, default values associated with each policy, and their integration into the Ktor framework as plugins.

% conclusion:
This project successfully developed
and deployed a dedicated Kotlin Multiplatform library that offers resilience mechanisms,
and includes a Ktor integration to provide an immediate installation and configuration of these mechanisms in asynchronous server and client architectures.

\textbf{Keywords:} Kotlin Multiplatform; library; fault-tolerance; resilience; distributed systems; Ktor framework
