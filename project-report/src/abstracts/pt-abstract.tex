\chapter*{Resumo}\label{ch:resumo}

% introdução:
Este documento aborda uma necessidade no ecossistema Kotlin Multiplataform,
ao desenvolver uma biblioteca de tolerância a falhas.
Kotlin Multiplatform é uma tecnologia que permite aos desenvolvidores partilhar código em várias plataformas, promovendo a reutilização e manutenção do mesmo.

% objetivo:
O objetivo principal é
criar uma biblioteca Kotlin Multiplataform
que ofereça mecanismos de resiliência (e.g., Retry, Circuit Breaker, Rate Limiter).
Estes mecanismos, sozinhos ou combinados, previnem ou mitigam as falhas inevitáveis que ocorrem em sistemas distribuídos,
que afectam a disponibilidade e a fiabilidade dos serviços.
Além disso, o projeto visa integrar estes mecanismos no Ktor,
uma framework Kotlin Multiplatform para a construção de serviços assíncronos de servidor e cliente.

% metodologia:
O projeto começou com uma revisão das soluções existentes em várias plataformas e linguagens para identificar padrões e práticas comuns, bem como diferenças e limitações.
Com este conhecimento, foi concebido um modelo para representar a interface comum para todos os mecanismos de resiliência implementados pela biblioteca - o Mechanism Model.
O projeto explorou vários mecanismos de resiliência,
explicando a sua funcionalidade,
detalhes de conceção e implementação escolhidos, capacidades de configuração, valores por defeito associados a cada política e a sua integração na framework Ktor como plugins.

% conclusão:
Este projeto desenvolveu com sucesso
e implantou
com sucesso uma biblioteca Kotlin Multiplatform que oferece mecanismos de resiliência,
e inclui uma integração Ktor para facilitar a instalação e configuração destes mecanismos em arquiteturas assíncronas de servidor e cliente.


\textbf{Palavras-chave:} Kotlin Multiplatform; biblioteca; tolerância a falhas; resiliência; sistemas distribuídos; framework Ktor
