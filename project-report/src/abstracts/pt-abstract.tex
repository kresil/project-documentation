\chapter*{Resumo}\label{ch:resumo}

% objetivo:
Este projeto aborda uma necessidade crítica no ecossistema Multiplatform Kotlin,
ao desenvolver uma biblioteca dedicada à tolerância a falhas.
Kotlin Multiplatform é uma tecnologia que permite aos programadores partilhar código em diversas plataformas, promovendo a reutilização e manutenção do código.

% objetivo:
O objetivo principal é
criar uma biblioteca Kotlin Multiplatform
que ofereça mecanismos de resiliência essenciais (e.g., Retry, Circuit Breaker, Rate Limiter)
que, sozinhos ou combinados, previnem ou mitigam as inevitáveis falhas que
ocorrem em sistemas distribuídos,
que afectam a disponibilidade e a fiabilidade dos serviços.
Além disso, o projeto visa integrar estes mecanismos no Ktor,
o único framework Multiplatform Kotlin para a construção de serviços assíncronos de servidor e cliente.

% metodologia:
O projeto começou com uma revisão das soluções existentes para mecanismos de resiliência em várias plataformas e linguagens
para identificar padrões e práticas comuns, bem como diferenças e limitações.
Com este conhecimento,
foi concebido um modelo para representar a interface comum a todos os mecanismos de resiliência implementados pela biblioteca - o Mechanism Model.
O projeto explorou então cada um dos
vários (mas não todos) mecanismos de resiliência,
detalhando a sua funcionalidade,
os pormenores de conceção e implementação escolhidos, as capacidades de configuração, os valores por defeito associados a cada política e a sua integração na framework Ktor como plugins.

% conclusão:
Uma vez que o projeto se encontra em fase de desenvolvimento, o trabalho futuro inclui
expandir a biblioteca com mecanismos de resiliência adicionais
e melhorar ainda mais os seus testes e documentação.
Desta forma, fornecer aos programadores um conjunto de ferramentas abrangente e fiável para a construção de sistemas distribuídos resilientes
em Kotlin Multiplatform.

\textbf{Palavras-chave:} Kotlin Multiplatform; biblioteca; tolerância a falhas; resiliência; sistemas distribuídos; framework Ktor
